\documentclass{article}
\usepackage[letterpaper,left=0.75in, right=0.75in, top=0.75in, bottom=0.75in]{geometry}
\usepackage{graphicx}
\usepackage[justification=centering]{subcaption}
\captionsetup[subfigure]{format=hang,justification=raggedright,singlelinecheck=false}
\usepackage{amsmath,amsthm,amsfonts,amssymb,mathtools, bm}
\usepackage[mmddyyyy]{datetime}
\graphicspath{{/home/jklimavicz/Documents/Plate_Analyzer/examples/stats/output/images/pdf}}
\setcounter{topnumber}{8}
\setcounter{bottomnumber}{8}
\setcounter{totalnumber}{8}
\usepackage{fancyhdr}
\pagestyle{fancy}
\fancyhf{}
\fancyhead[L]{Merlin Bioassay Results}
\fancyfoot[C]{\thepage}
\fancyhead[R]{Compiled on \today}
\renewcommand{\headrulewidth}{0pt}
\DeclareMathOperator*{\argmin}{argmin}
\DeclareMathOperator*{\argmax}{argmax}



%%%%%%%%%%%%%%%%%%%%%%%%%%%%%%%%%%%%%%%%%%%%%%%%%%%%%%%%%%%%%%%%\title{Merlin Bioassay Results}
\author{James Klimavicz}
\date{\today}
\begin{document}


\begin{figure}[thp!]
   \begin{subfigure}{0.500\textwidth}
      \centering
      \includegraphics[width = {0.95\textwidth}]{/home/jklimavicz/Documents/Plate_Analyzer/examples/stats/output/images/pdf/acetamiprid.pdf}
      \vspace{-0.05cm}
      \caption*{\textbf{Acetamiprid} LC$_{50}$: 1.39 ppm [0.992, 1.91] \\ 
4 biol. reps; 5 tech. reps; R$^2$: 0.82}
      \vspace{0.1cm}
   \end{subfigure}%
   \begin{subfigure}{0.500\textwidth}
      \centering
      \includegraphics[width = {0.95\textwidth}]{/home/jklimavicz/Documents/Plate_Analyzer/examples/stats/output/images/pdf/aldicarb.pdf}
      \vspace{-0.05cm}
      \caption*{\textbf{Aldicarb} LC$_{50}$: 0.526 ppm [0.424, 0.645] \\ 
4 biol. reps; 5 tech. reps; R$^2$: 0.941}
      \vspace{0.1cm}
   \end{subfigure}%
\vspace{-0.1cm}
   \begin{subfigure}{0.500\textwidth}
      \centering
      \includegraphics[width = {0.95\textwidth}]{/home/jklimavicz/Documents/Plate_Analyzer/examples/stats/output/images/pdf/aziphos-methyl.pdf}
      \vspace{-0.05cm}
      \caption*{\textbf{Aziphos-methyl} LC$_{50}$: 0.134 ppm [0.101, 0.176] \\ 
4 biol. reps; 5 tech. reps; R$^2$: 0.763}
      \vspace{0.1cm}
   \end{subfigure}%
   \begin{subfigure}{0.500\textwidth}
      \centering
      \includegraphics[width = {0.95\textwidth}]{/home/jklimavicz/Documents/Plate_Analyzer/examples/stats/output/images/pdf/bifenthrin.pdf}
      \vspace{-0.05cm}
      \caption*{\textbf{Bifenthrin} LC$_{50}$: 1.52 ppm [1.2, 1.94] \\ 
7 biol. reps; 8 tech. reps; R$^2$: 0.764}
      \vspace{0.1cm}
   \end{subfigure}%
\vspace{-0.1cm}
   \begin{subfigure}{0.500\textwidth}
      \centering
      \includegraphics[width = {0.95\textwidth}]{/home/jklimavicz/Documents/Plate_Analyzer/examples/stats/output/images/pdf/carbaryl.pdf}
      \vspace{-0.05cm}
      \caption*{\textbf{Carbaryl} LC$_{50}$: 778 ppm [42, 2.29e4] \\ 
8 biol. reps; 9 tech. reps; R$^2$: 0.0779}
      \vspace{0.1cm}
   \end{subfigure}%
   \begin{subfigure}{0.500\textwidth}
      \centering
      \includegraphics[width = {0.95\textwidth}]{/home/jklimavicz/Documents/Plate_Analyzer/examples/stats/output/images/pdf/carbofuran.pdf}
      \vspace{-0.05cm}
      \caption*{\textbf{Carbofuran} LC$_{50}$: 0.263 ppm [0.203, 0.335] \\ 
4 biol. reps; 5 tech. reps; R$^2$: 0.888}
      \vspace{0.1cm}
   \end{subfigure}%
\end{figure}
\clearpage
\pagebreak
\vspace{-0.1cm}
\begin{figure}[thp!]
   \begin{subfigure}{0.500\textwidth}
      \centering
      \includegraphics[width = {0.95\textwidth}]{/home/jklimavicz/Documents/Plate_Analyzer/examples/stats/output/images/pdf/chlorantraniliprole.pdf}
      \vspace{-0.05cm}
      \caption*{\textbf{Chlorantraniliprole} LC$_{50}$: 5.75 ppm [3.86, 9.2] \\ 
9 biol. reps; 10 tech. reps; R$^2$: 0.541}
      \vspace{0.1cm}
   \end{subfigure}%
   \begin{subfigure}{0.500\textwidth}
      \centering
      \includegraphics[width = {0.95\textwidth}]{/home/jklimavicz/Documents/Plate_Analyzer/examples/stats/output/images/pdf/chlorfenapyr.pdf}
      \vspace{-0.05cm}
      \caption*{\textbf{Chlorfenapyr} LC$_{50}$: 10.4 ppm [8.63, 12.6] \\ 
4 biol. reps; 5 tech. reps; R$^2$: 0.869}
      \vspace{0.1cm}
   \end{subfigure}%
\vspace{-0.1cm}
   \begin{subfigure}{0.500\textwidth}
      \centering
      \includegraphics[width = {0.95\textwidth}]{/home/jklimavicz/Documents/Plate_Analyzer/examples/stats/output/images/pdf/chlorpyrifos.pdf}
      \vspace{-0.05cm}
      \caption*{\textbf{Chlorpyrifos} LC$_{50}$: 0.196 ppm [0.154, 0.243] \\ 
4 biol. reps; 5 tech. reps; R$^2$: 0.923}
      \vspace{0.1cm}
   \end{subfigure}%
   \begin{subfigure}{0.500\textwidth}
      \centering
      \includegraphics[width = {0.95\textwidth}]{/home/jklimavicz/Documents/Plate_Analyzer/examples/stats/output/images/pdf/clothianidin.pdf}
      \vspace{-0.05cm}
      \caption*{\textbf{Clothianidin} LC$_{50}$: 0.824 ppm [0.543, 1.23] \\ 
4 biol. reps; 5 tech. reps; R$^2$: 0.693}
      \vspace{0.1cm}
   \end{subfigure}%
\vspace{-0.1cm}
   \begin{subfigure}{0.500\textwidth}
      \centering
      \includegraphics[width = {0.95\textwidth}]{/home/jklimavicz/Documents/Plate_Analyzer/examples/stats/output/images/pdf/cyantraniliprole.pdf}
      \vspace{-0.05cm}
      \caption*{\textbf{Cyantraniliprole} LC$_{50}$: 0.828 ppm [0.613, 1.13] \\ 
9 biol. reps; 10 tech. reps; R$^2$: 0.801}
      \vspace{0.1cm}
   \end{subfigure}%
   \begin{subfigure}{0.500\textwidth}
      \centering
      \includegraphics[width = {0.95\textwidth}]{/home/jklimavicz/Documents/Plate_Analyzer/examples/stats/output/images/pdf/beta-cypermethrin.pdf}
      \vspace{-0.05cm}
      \caption*{\textbf{$\beta$-Cypermethrin} LC$_{50}$: 0.112 ppm [0.0777, 0.157] \\ 
7 biol. reps; 8 tech. reps; R$^2$: 0.437}
      \vspace{0.1cm}
   \end{subfigure}%
\end{figure}
\clearpage
\pagebreak
\vspace{-0.1cm}
\begin{figure}[thp!]
   \begin{subfigure}{0.500\textwidth}
      \centering
      \includegraphics[width = {0.95\textwidth}]{/home/jklimavicz/Documents/Plate_Analyzer/examples/stats/output/images/pdf/ddt.pdf}
      \vspace{-0.05cm}
      \caption*{\textbf{DDT} LC$_{50}$: 117 ppm [10.2, 9.12e3] \\ 
4 biol. reps; 5 tech. reps; R$^2$: 0.0807}
      \vspace{0.1cm}
   \end{subfigure}%
   \begin{subfigure}{0.500\textwidth}
      \centering
      \includegraphics[width = {0.95\textwidth}]{/home/jklimavicz/Documents/Plate_Analyzer/examples/stats/output/images/pdf/ddvp.pdf}
      \vspace{-0.05cm}
      \caption*{\textbf{DDVP} LC$_{50}$: 0.0445 ppm [0.0282, 0.065] \\ 
5 biol. reps; 6 tech. reps; R$^2$: 0.469}
      \vspace{0.1cm}
   \end{subfigure}%
\vspace{-0.1cm}
   \begin{subfigure}{0.500\textwidth}
      \centering
      \includegraphics[width = {0.95\textwidth}]{/home/jklimavicz/Documents/Plate_Analyzer/examples/stats/output/images/pdf/fipronil.pdf}
      \vspace{-0.05cm}
      \caption*{\textbf{Fipronil} LC$_{50}$: 1.8 ppm [1.48, 2.23] \\ 
5 biol. reps; 6 tech. reps; R$^2$: 0.676}
      \vspace{0.1cm}
   \end{subfigure}%
   \begin{subfigure}{0.500\textwidth}
      \centering
      \includegraphics[width = {0.95\textwidth}]{/home/jklimavicz/Documents/Plate_Analyzer/examples/stats/output/images/pdf/imidacloprid.pdf}
      \vspace{-0.05cm}
      \caption*{\textbf{Imidacloprid} LC$_{50}$: 0.562 ppm [0.403, 0.786] \\ 
9 biol. reps; 11 tech. reps; R$^2$: 0.776}
      \vspace{0.1cm}
   \end{subfigure}%
\vspace{-0.1cm}
   \begin{subfigure}{0.500\textwidth}
      \centering
      \includegraphics[width = {0.95\textwidth}]{/home/jklimavicz/Documents/Plate_Analyzer/examples/stats/output/images/pdf/malathion.pdf}
      \vspace{-0.05cm}
      \caption*{\textbf{Malathion} LC$_{50}$: 0.362 ppm [0.33, 0.396] \\ 
16 biol. reps; 31 tech. reps; R$^2$: 0.789}
      \vspace{0.1cm}
   \end{subfigure}%
   \begin{subfigure}{0.500\textwidth}
      \centering
      \includegraphics[width = {0.95\textwidth}]{/home/jklimavicz/Documents/Plate_Analyzer/examples/stats/output/images/pdf/naled.pdf}
      \vspace{-0.05cm}
      \caption*{\textbf{Naled} LC$_{50}$: 0.0486 ppm [0.0321, 0.0718] \\ 
7 biol. reps; 7 tech. reps; R$^2$: 0.566}
      \vspace{0.1cm}
   \end{subfigure}%
\end{figure}
\clearpage
\pagebreak
\vspace{-0.1cm}
\begin{figure}[thp!]
   \begin{subfigure}{0.500\textwidth}
      \centering
      \includegraphics[width = {0.95\textwidth}]{/home/jklimavicz/Documents/Plate_Analyzer/examples/stats/output/images/pdf/naphazoline hydrochloride.pdf}
      \vspace{-0.05cm}
      \caption*{\textbf{Naphazoline hydrochloride} LC$_{50}$: 1.02e3 ppm [74.1, 3.07e17] \\ 
2 biol. reps; 2 tech. reps; R$^2$: 0.0569}
      \vspace{0.1cm}
   \end{subfigure}%
   \begin{subfigure}{0.500\textwidth}
      \centering
      \includegraphics[width = {0.95\textwidth}]{/home/jklimavicz/Documents/Plate_Analyzer/examples/stats/output/images/pdf/s-nicotine.pdf}
      \vspace{-0.05cm}
      \caption*{\textbf{\textit{S}-nicotine} LC$_{50}$: 129 ppm [52, 4.05e3] \\ 
4 biol. reps; 5 tech. reps; R$^2$: 0.272}
      \vspace{0.1cm}
   \end{subfigure}%
\vspace{-0.1cm}
   \begin{subfigure}{0.500\textwidth}
      \centering
      \includegraphics[width = {0.95\textwidth}]{/home/jklimavicz/Documents/Plate_Analyzer/examples/stats/output/images/pdf/permethrin.pdf}
      \vspace{-0.05cm}
      \caption*{\textbf{Permethrin} LC$_{50}$: 11.9 ppm [7.42, 36.3] \\ 
5 biol. reps; 6 tech. reps; R$^2$: 0.225}
      \vspace{0.1cm}
   \end{subfigure}%
   \begin{subfigure}{0.500\textwidth}
      \centering
      \includegraphics[width = {0.95\textwidth}]{/home/jklimavicz/Documents/Plate_Analyzer/examples/stats/output/images/pdf/phorate.pdf}
      \vspace{-0.05cm}
      \caption*{\textbf{Phorate} LC$_{50}$: 0.225 ppm [0.165, 0.295] \\ 
4 biol. reps; 5 tech. reps; R$^2$: 0.897}
      \vspace{0.1cm}
   \end{subfigure}%
\vspace{-0.1cm}
   \begin{subfigure}{0.500\textwidth}
      \centering
      \includegraphics[width = {0.95\textwidth}]{/home/jklimavicz/Documents/Plate_Analyzer/examples/stats/output/images/pdf/propoxur.pdf}
      \vspace{-0.05cm}
      \caption*{\textbf{Propoxur} LC$_{50}$: 1.39 ppm [1.12, 1.74] \\ 
6 biol. reps; 7 tech. reps; R$^2$: 0.824}
      \vspace{0.1cm}
   \end{subfigure}%
   \begin{subfigure}{0.500\textwidth}
      \centering
      \includegraphics[width = {0.95\textwidth}]{/home/jklimavicz/Documents/Plate_Analyzer/examples/stats/output/images/pdf/pyethrum.pdf}
      \vspace{-0.05cm}
      \caption*{\textbf{Pyethrum} LC$_{50}$: 74.1 ppm [38.5, 189] \\ 
8 biol. reps; 9 tech. reps; R$^2$: 0.226}
      \vspace{0.1cm}
   \end{subfigure}%
\end{figure}
\clearpage
\pagebreak
\vspace{-0.1cm}
\pagebreak


Data analysis was performed using the statistics module for the Merlin Data Analysis program. 
Live/dead counts from the bioassay were used to generate new survival probabilities using a Beta prior. The user-specified prior is Heldane's prior, the improper prior $\text{Beta}(0,0)$, (set by \texttt{BETA\_PRIOR}) and 3840 bootstrap iterations were used (set by \texttt{BOOTSTRAP\_ITERS}).When either the live count or dead count was equal to 0, the prior the distribution $\text{Beta}(0.25,0.25)$ (set by \texttt{BETA\_PRIOR\_0}) was used to avoid the sunrise problem. Correlation between wells in a replicate was modelled by generating multivariate normal random variables with correlation $\rho = 0.1$ (set by \texttt{RHO}),  which were then converted to quantiles, and then back-converted to probabilities in the appropriate beta distribution. 

Each iteration of bootstrapped dose-response data was fit to the curve $$\theta_i = \frac{b_2}{\left(1 + \exp(b_0+b_1x_i) \right)}$$ by maximizing the log-likelihood function, \textit{i.e.}, solving 
\begin{align*}
&\argmax_{\bm b = (b_0, b_1, b_2)} \left( f(\bm b) + \sum^{n}_{i=1} (1-\theta_i) \ln(1 + \xi_i - b_2) + \sum^{n}_{i=1} \theta_i\ln b_2 - \sum^{n}_{i=1} \ln (1 + \xi_i)\right)\\\intertext{where} 
 \qquad \qquad f(\bm b) &= \frac{-\left(b_0^2 + b_1^2\right)}{2\sigma^2} + (\alpha - 1) \ln b_2 + (\beta - 1)\ln(1-b_2)\\\intertext{and} 
\qquad \qquad \xi_i &= e^{b_0 + b_1 x_i}.\end{align*}
Priors on parameters were $(b_0, b_1) \sim \mathcal{N}(\bm 0,\sigma \mc I_2)$and $b_2 \sim \text{Beta}(\alpha, \beta)$, where $\sigma = 1000.0$, $\alpha = 1.5$, and $\beta = 1.001$, as defined by \texttt{LL\_SIGMA}, \texttt{LL\_BETA1}, and \texttt{LL\_BETA2} in the \texttt{analysis\_config.txt} file, respectively.  Optimization of the log-likelihood function was performed using the optimized vector Broyden-Fletcher-Goldfarb-Shanno (BFGS) algorithm using a C interface to the GNU Scientific Library. 

Credible intervals for the data points are shown at the 80\% level when fewer than 10 replicates are used. The best-fit line is calculated as the median value of all fitted curves at a given concentration. The error region for the curve respresents a 95\% confidence region, as determined by quantiles of predicted survivals at each concentration.


\end{document}